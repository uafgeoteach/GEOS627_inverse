% dvips -t letter lab_svdgeom.dvi -o lab_svdgeom.ps ; ps2pdf lab_svdgeom.ps
\documentclass[11pt,titlepage,fleqn]{article}

\usepackage{amsmath}
\usepackage{amssymb}
\usepackage{latexsym}
\usepackage[round]{natbib}
%\usepackage{epsfig}
\usepackage{graphicx}
\usepackage{bm}

\usepackage{url}
\usepackage{color}

% careful: there could be conflicts when using the xr package
\usepackage{hyperref}

%--------------------------------------------------------------
%       SPACING COMMANDS (Latex Companion, p. 52)
%--------------------------------------------------------------

\usepackage{setspace}    % double-space or single-space
\usepackage{xspace}

\renewcommand{\baselinestretch}{1.2}

\textwidth 460pt
\textheight 690pt
\oddsidemargin 0pt
\evensidemargin 0pt

% see Latex Companion, p. 85
\voffset     -50pt
\topmargin     0pt
\headsep      20pt
\headheight   15pt
\headheight    0pt
\footskip     30pt
\hoffset       0pt

%\input{carlcommands}
\input{commands_letters}
\input{commands_uaf}
\input{commands_carl}

\newcommand{\blank}{xxxx}

\newcommand{\cyear}{2027}

% provide space for students to write their solutions
\newcommand{\vertgap}{\vspace{1cm}}

\graphicspath{
  {./figures/}
}

\newcommand{\repodir}{{\tt inverse}}

\newcommand{\cltag}{GEOS 627/427: Inverse Problems and Parameter Estimation, Carl Tape}
\newcommand{\ptag}{{\bf \textcolor{magenta}{[GEOS 627]}}}
%\newcommand{\ptag}{}

\bibliographystyle{agufull08}

\newcommand{\howmuchtime}{Approximately how much time {\em outside of class and lab time} did you spend on this problem set? Feel free to suggest improvements here.}

%------------------------------------------------------

\newcommand{\Ucolor}{\textcolor{red}{\bU}}
\newcommand{\Vcolor}{\textcolor{blue}{\bV}}

\newcommand{\Gcolor}{\textcolor{red}{\bU}\bS\textcolor{blue}{\bV^T}}
\newcommand{\Gpcolor}{\textcolor{red}{\bU_p}\,\bS_p\textcolor{blue}{\bV_p^T}}
\newcommand{\Gdcolor}{\textcolor{blue}{\bV_p}\,\bS_p^{-1}\textcolor{red}{\bU_p^T}}
\newcommand{\GcolorT}{\textcolor{blue}{\bV}\bS^T\textcolor{red}{\bU^T}}
\newcommand{\GpcolorT}{\textcolor{blue}{\bV_p}\,\bS_p\textcolor{red}{\bU_p^T}}
\newcommand{\GdcolorT}{\textcolor{red}{\bU_p}\,\bS_p^{-1}\textcolor{blue}{\bV_p^T}}

%------------------------------------------------------


%\renewcommand{\baselinestretch}{1.1}

\newcommand{\tfile}{{\tt lab\_svdgeometry.ipynb}}
\newcommand{\nfile}{{\tt notes\_svd.pdf}}

% change the figures to ``Figure L3'', etc
\renewcommand{\thefigure}{L\arabic{figure}}
\renewcommand{\thetable}{L\arabic{table}}
\renewcommand{\theequation}{L\arabic{equation}}
\renewcommand{\thesection}{L\arabic{section}}

%--------------------------------------------------------------
\begin{document}
%-------------------------------------------------------------

\begin{spacing}{1.2}
\centering
{\large \bf Lab Exercise: Geometry of the singular value decomposition [svdgeometry]} \\
\cltag\ \\
Last compiled: \today
\end{spacing}


%------------------------

\subsection*{Overview}

%The singular value decomposition is featured in \citet{Aster}.
The objective of this lab is to reinforce the geometrical concept of SVD \citep{TrefethenBau} presented in \nfile. You will be making measurements on paper; these will benefit from having a ruler and calculator. Main goals:
%
\begin{enumerate}
\item Determine $\bG$ from how a unit circle (in the model space) is distorted into an ellipse (in the data space).
\item Depict how a unit circle (in the data space) is distorted into an ellipse (in the model space) by $\bG^{\dagger}$.
\end{enumerate}

%------------------------

\subsection*{Part 1: $\bG$}

This lab includes four examples of ($2 \times 2$) $\bG$ representing the SVD (\refFig{fig:2D}). \\
Four additional examples (Examples 5--8) are shown in Figure~1 of \nfile.

\begin{enumerate}

\item Start with \refFig{fig:ex1}.
%
\begin{enumerate}
\item Determine the singular values $s_1$ and $s_2$. Note that if $s_1 = s_2 = 1$, then the map of the unit circle would be unchanged.

\item Determine $\Vcolor$ (\textcolor{blue}{model space}).

Hint: Use a ruler, then make sure that your vectors are normalized. See Note \footnote{
Measuring the coordinates of two basis vectors will not guarantee that the two vectors are orthogonal. For this, you could use a rotation matrix
%
\begin{equation}
\bR_\theta = \left[
\begin{array}{rr}
\cos\theta & -\sin\theta \\ 
\sin\theta & \cos\theta
\end{array}
\right]
\end{equation}
%
This can be implemented as a function. Provide a $\theta$, then the columns of $\bR_\theta$ will be the rotated standard basis vectors.

To get $\Ucolor$ and $\Vcolor$, use $\bR_\theta$. You may also want to use
%
\begin{equation}
\bZ = \left[
\begin{array}{rr}
0 & 1 \\ 
1 & 0
\end{array}
\right],
\end{equation}
%
since $\bA\bZ$ will swap the two column vectors of $\bA$.
}.

\item Determine $\Ucolor$ (\textcolor{red}{data space}).

\end{enumerate}

\item Calculate $\bG$. It is probably easiest to use the template notebook \tfile.

If your entries of $\bG$ are not close (within 0.2) to integer values, please try again.

\item Make sure you have the true $\bG$ before proceeding.

\item In the {\bf top} of \refFig{fig:ex1}, sketch $\be_1$ and $\be_2$ in the left plot. \\
Sketch $\bG\be_1$ and $\bG\be_2$ in the right plot.

\item An example model
%
\begin{eqnarray}
\bem  &=& \left[ \begin{array}{c} -1 \\ -1 \end{array} \right]
\end{eqnarray}
%
is plotted. Calculate $\bd = \bG\bem$ by hand and plot it at right.

\item By hand, calculate the eigenbasis $\bH$ of $\bG$ (see \verb+notes_matrix.pdf+):
%
\begin{eqnarray}
\bG\bH  &=& \bH \bD
\\
\bH &=& [\bh_1\;\bh_2]
\\
\bD &=& \left[ \begin{array}{rr}
     \lambda_1  &  0  \\
     0  &  \lambda_2  \\
\end{array} \right]
\\
\bG\bh_1 &=& \lambda_1\bh_1
\\
\bG\bh_2 &=& \lambda_2\bh_2
\end{eqnarray}
%
\begin{itemize}
\item Sketch $\bh_1$ and $\bh_2$ in the model space.
\item Are $\bh_1$ and $\bh_2$ orthogonal?
\item Sketch $\lambda_1\bh_1$ and $\lambda_2\bh_2$ in the data space.
\end{itemize}

\item Repeat the above steps to determine the three other $\bG$ matrices in \refFigab{fig:ex2}{fig:ex4}. Annotate the plots as requested above.

\item Try the other four examples in \refFigab{fig:ex5}{fig:ex8}.

Note: You can't sketch the eigenvectors if they are complex.

\end{enumerate}

%=================================================================

%------------------------

\subsection*{Part 2: $\bG^{\dagger}$}

\begin{enumerate}
\item What is the analagous set of equations to (13)--(15) in \nfile, for $\bG^{\dagger}$ instead of $\bG$? What do they mean, in words?

(Hint: What are $\nparm$, $\ndata$, and $p$ in our case?)

\item The {\bf bottom} of \refFig{fig:ex1} represents the mapping $\bG^{\dagger}\Ucolor$. Sketch the following (note that you do not need $\bG^{\dagger}$ for the sketch):
%
\begin{itemize}
\item $\bu_1$ and $\bu_2$ in the left plot
\item $\bv_1$ and $\bv_2$ in the right plot
\item appropriately scaled versions of $\bv_1$ and $\bv_2$ in the right plot
\item the ellipse representing $\bG^{\dagger}\Ucolor$ in the right plot
\end{itemize}

\item Open \tfile. What does the function \verb+svdmat()+ do (see \verb+lib_inverse.py+)?

\item Using \verb+svdmat+, calculate $\Ucolor$, $\Vcolor$, $\bS$, and $\bG^{\dagger}$. \\
Sketch $\be_1$ and $\be_2$ in the lower left plot. \\
Sketch $\bG^{\dagger}\be_1$ and $\bG^{\dagger}\be_2$ in the lower right plot.

\item Does $\bG^{\dagger} = \bG^{-1}$? Why?

\item Get some more practice with three more examples in \refFigab{fig:ex2}{fig:ex4}. For each case, calculate the SVD of $\bG$, then $\bG^{\dagger}$, and sketch the following in the lower plots:
%
%
\begin{itemize}
\item $\bu_1$ and $\bu_2$ in the left plot
\item appropriately scaled versions of $\bv_1$ and $\bv_2$ in the right plot
\item the ellipse representing $\bG^{\dagger}\Ucolor$ in the right plot
\item $\be_1$ and $\be_2$ in the lower left plot.
\item $\bG^{\dagger}\be_1$ and $\bG^{\dagger}\be_2$ in the lower right plot.
\end{itemize}
%
If you are feeling ambitious, then try \refFigab{fig:ex5}{fig:ex8} as well.

\item Uncomment the break line in \tfile, then run the notebook.

By default, the code will generate a random $\bG$ having integer entries between $-2$ and 2.

Try a few different $\bG$ to see what happens.

\item The top of \refFig{fig:ex1} represents $\Vcolor \rightarrow \bG \rightarrow \Ucolor$.

Write some code to generate a plot representing $\Ucolor \rightarrow \bG^{\dagger} \rightarrow \Vcolor$. Stick to the coloring convention for $\Ucolor$ and $\Vcolor$.

\end{enumerate}

%-------------------------------------------------------------
\bibliography{uaf_abbrev,uaf_main}
%-------------------------------------------------------------

\setcounter{figure}{-1}

\clearpage\pagebreak
\pagestyle{empty}
\begin{figure}
\centering
\begin{tabular}{rcc}
Ex 1 & $........$ & \includegraphics[width=11cm]{svd_2D_1} \\ \\ \\
Ex 2 & &\includegraphics[width=11cm]{svd_2D_2} \\ \\ \\
Ex 3 & &\includegraphics[width=11cm]{svd_2D_3} \\ \\ \\
Ex 4 & &\includegraphics[width=11cm]{svd_2D_4} 
\end{tabular}
\caption[]
{{
Examples 1--4.
The matrix $\bG = \Ucolor\,\bS\,\Vcolor^T$ transforms the orthonormal basis vectors \textcolor{blue}{$\{\bv_i\}$} to the orthogonal vectors \textcolor{red}{$\{s_i\bu_i\}$}: \makebox{$\bG\Vcolor = \Ucolor\,\bS$}.
\label{fig:2D}
}}
\end{figure}

\clearpage\pagebreak
\begin{figure}
\hspace{-1cm}
\includegraphics[width=18cm]{svd_2D_both_1}
\caption[]
{{
Example 1.
\label{fig:ex1}
}}
\end{figure}

\clearpage\pagebreak
\begin{figure}
\hspace{-1cm}
\includegraphics[width=18cm]{svd_2D_both_2}
\caption[]
{{
Example 2.
\label{fig:ex2}
}}
\end{figure}

\clearpage\pagebreak
\begin{figure}
\hspace{-1cm}
\includegraphics[width=18cm]{svd_2D_both_3}
\caption[]
{{
Example 3.
\label{fig:ex3}
}}
\end{figure}

\clearpage\pagebreak
\begin{figure}
\hspace{-1cm}
\includegraphics[width=18cm]{svd_2D_both_4}
\caption[]
{{
Example 4.
\label{fig:ex4}
}}
\end{figure}

\clearpage\pagebreak
\begin{figure}
\hspace{-1cm}
\includegraphics[width=18cm]{svd_2D_both_5}
\caption[]
{{
Example 5.
\label{fig:ex5}
}}
\end{figure}

\clearpage\pagebreak
\begin{figure}
\hspace{-1cm}
\includegraphics[width=18cm]{svd_2D_both_6}
\caption[]
{{
Example 6.
\label{fig:ex6}
}}
\end{figure}

\clearpage\pagebreak
\begin{figure}
\hspace{-1cm}
\includegraphics[width=18cm]{svd_2D_both_7}
\caption[]
{{
Example 7.
\label{fig:ex7}
}}
\end{figure}

\clearpage\pagebreak
\begin{figure}
\hspace{-1cm}
\includegraphics[width=18cm]{svd_2D_both_8}
\caption[]
{{
Example 8.
\label{fig:ex8}
}}
\end{figure}

%-------------------------------------------------------------
\end{document}
%-------------------------------------------------------------
