% dvips -t letter hw_ch1.dvi -o hw_ch1.ps ; ps2pdf hw_ch1.ps
\documentclass[11pt,titlepage,fleqn]{article}

\usepackage{amsmath}
\usepackage{amssymb}
\usepackage{latexsym}
\usepackage[round]{natbib}
%\usepackage{epsfig}
\usepackage{graphicx}
\usepackage{bm}

\usepackage{url}
\usepackage{color}

% careful: there could be conflicts when using the xr package
\usepackage{hyperref}

%--------------------------------------------------------------
%       SPACING COMMANDS (Latex Companion, p. 52)
%--------------------------------------------------------------

\usepackage{setspace}    % double-space or single-space
\usepackage{xspace}

\renewcommand{\baselinestretch}{1.2}

\textwidth 460pt
\textheight 690pt
\oddsidemargin 0pt
\evensidemargin 0pt

% see Latex Companion, p. 85
\voffset     -50pt
\topmargin     0pt
\headsep      20pt
\headheight   15pt
\headheight    0pt
\footskip     30pt
\hoffset       0pt

%\input{carlcommands}
\input{commands_letters}
\input{commands_uaf}
\input{commands_carl}

\newcommand{\blank}{xxxx}

\newcommand{\cyear}{2027}

% provide space for students to write their solutions
\newcommand{\vertgap}{\vspace{1cm}}

\graphicspath{
  {./figures/}
}

\newcommand{\repodir}{{\tt inverse}}

\newcommand{\cltag}{GEOS 627/427: Inverse Problems and Parameter Estimation, Carl Tape}
\newcommand{\ptag}{{\bf \textcolor{magenta}{[GEOS 627]}}}
%\newcommand{\ptag}{}

\bibliographystyle{agufull08}

\newcommand{\howmuchtime}{Approximately how much time {\em outside of class and lab time} did you spend on this problem set? Feel free to suggest improvements here.}

%------------------------------------------------------

\newcommand{\Ucolor}{\textcolor{red}{\bU}}
\newcommand{\Vcolor}{\textcolor{blue}{\bV}}

\newcommand{\Gcolor}{\textcolor{red}{\bU}\bS\textcolor{blue}{\bV^T}}
\newcommand{\Gpcolor}{\textcolor{red}{\bU_p}\,\bS_p\textcolor{blue}{\bV_p^T}}
\newcommand{\Gdcolor}{\textcolor{blue}{\bV_p}\,\bS_p^{-1}\textcolor{red}{\bU_p^T}}
\newcommand{\GcolorT}{\textcolor{blue}{\bV}\bS^T\textcolor{red}{\bU^T}}
\newcommand{\GpcolorT}{\textcolor{blue}{\bV_p}\,\bS_p\textcolor{red}{\bU_p^T}}
\newcommand{\GdcolorT}{\textcolor{red}{\bU_p}\,\bS_p^{-1}\textcolor{blue}{\bV_p^T}}

%------------------------------------------------------


\renewcommand{\baselinestretch}{1.1}

%--------------------------------------------------------------
\begin{document}
%-------------------------------------------------------------

\begin{spacing}{1.2}
\centering
{\large \bf Problem Set 2: Aster Ch. 1 [ch1]} \\
\cltag\ \\
Assigned: January 24, \cyear\ --- Due: January 31, \cyear\ \\
Last compiled: \today
\end{spacing}

%------------------------

\subsection*{Instructions}

Review Appendix A of \citet{Aster}. In your solutions, be sure to explain your steps. Be clear to distinguish scalars ($k$), vectors ($\bv$), matrices ($\bG$), and operators ($G$). If your answers are hand-written, you might want to put single bars under vectors and double bars under matrices.

%\pagebreak
\subsection*{Problem 1 (1.0). Aster 1-1.}

Consider a mathematical model of the form $G(\bem) = \bd$, where $\bem$ is a vector of length $\nparm$, and $\bd$ is a vector of length $\ndata$. Suppose that the model obeys the superposition and scaling laws and is thus linear. Show that $G(\bem)$ can be written in the form
%
\begin{equation}
G(\bem) = \bG\bem
\end{equation}
%
where $\bG$ is an $\ndata \times \nparm$ matrix.

\medskip\noindent
Hints:
%
\begin{itemize}
\item Ignore $\bd$.
\item Consider the standard basis, and write $\bem$ as a linear combination of the vectors in the standard basis.
\item Apply the superposition and scaling laws.
\item Finally, recall the definition of matrix-vector multiplication.
\end{itemize}

%--------------------------------------------------------------

\clearpage\pagebreak
\subsection*{Problem 2 (2.0). Aster 1-2 modified.}

Consider the ballistic problem in Section 1.2.
The mathematical model is \citep[][Eq.~1.13]{Aster}
%
\begin{equation}
y(t) = m_1 + m_2 t - \sfrac{1}{2} m_3 t^2.
\label{ballistic}
\end{equation}
%
Let us assume that $\byh$ points upward.
% such that a positive value of $m_3$ will pull the ballistic downward.
The corresponding system of equations is
%
\begin{eqnarray}
\bG \bem &=& \dvec
\label{lsq}
\\
\left[
\begin{array}{ccc}
1 & t_1 & -(1/2)t_1^2 \\
1 & t_2 & -(1/2)t_2^2 \\
1 & t_3 & -(1/2)t_3^2 \\
\vdots & \vdots & \vdots \\
1 & t_\ndata & -(1/2)t_\ndata^2 \\
\end{array}
\right]
\left[
\begin{array}{c}
m_1 \\ m_2 \\ m_3
\end{array}
\right]
&=& 
\left[
\begin{array}{c}
y_1 \\ y_2 \\ y_3 \\ \vdots \\ y_\ndata
\end{array}
\right]
\end{eqnarray}
%
Let $\ndata = \nparm = 3$ in this problem: {\bf there are $\ndata=3$ people, and each person makes one measurement $(t_i,y_i)$ of the height $y_i$ of the ball at time $t_i$.} Assumptions:
%
\begin{itemize}
\item A ball is being shot from elevation $m_1$ (say, from the middle floor of a high-rise building) with vertical velocity $m_2$ in an environment with vertical acceleration given by $m_3$.
\item The influence of the ball's height on acceleration is negligible.
\item Realistic influences such as air friction are negligible.
\item Each user's measurement $(t_i,y_i)$ may have error (\ie is not accurate).
\end{itemize}
%
After the experiment, the three measurements are tabulated.
Your task is to solve for the $\nparm=3$ unknowns $m_1, m_2, m_3$.
%
\begin{enumerate}
\item (0.8) Describe the physical meaning of $m_1 < 0$, $m_1 = 0$, $m_1 > 0$.

Describe the physical meaning\footnote{``The ball moves in the direction of $-\byh$'' is {\em not} what I mean by ``physical meaning.''} of $m_2 < 0$, $m_2 = 0$, $m_2 > 0$.

Describe the physical meaning of $m_3 < 0$, $m_3 = 0$, $m_3 > 0$.

\item (0.4) Under what circumstances (if any) will there be a {\bf single solution}?

Plot or sketch an example in $y$-$t$ space that contains three observations and (if it exists) the $\bem$ fitting model(s).

\item (0.4) Under what circumstances (if any) will there be {\bf no solution}?

Plot or sketch an example, as before.

\item (0.4) Under what circumstances (if any) will there be {\bf multiple solutions}?

Plot or sketch an example, as before.
\end{enumerate}
%
Hint: Row-reducing the $3 \times 3$ system of equations (perhaps with an online tool) may provide insights, though you do not need this in order to answer the question completely. 
% THIS WAS DONE IN MATHEMATICA
In order to row-reduce the augmented matrix, I typed the following command into the online Mathematica platform {\em Wolfram Development Platform}: \\
\verb+Simplify[MatrixForm[RowReduce[({{1,t1,-t1^2/2,y1},{1,t2,-t2^2/2,y2},{1,t3,-t3^2/2,y3}})]]]+
and got the following result:
%
\begin{eqnarray}
&& RREF \left[
\begin{array}{cccc}
1 & t_1 & -(1/2)t_1^2 & y_1 \\
1 & t_2 & -(1/2)t_2^2 & y_2 \\
1 & t_3 & -(1/2)t_3^2 & y_3 \\
\end{array}
\right]
\nonumber \\ \nonumber \\
&& =
\left[
\begin{array}{cccc}
1 & 0 & 0 & D^{-1}\left[t_1 t_3(t_3-t_1)y_2 + t_2^2(t_3 y_1 - t_1 y_3) + t_2(t_1^2 y_3 - t_3^2 y_1) \right] \\
& & & \\
0 & 1 & 0 & D^{-1}\left[ t_1^2(y_2-y_3) + t_2^2(y_3 -y_1) + t_3^2(y_1-y_2)\right] \\
& & & \\
0 & 0 & 1 & 2D^{-1}\left[ t_1(y_2-y_3) + t_2(y_3 -y_1) + t_3(y_1-y_2)\right] \\
\end{array}
\right]
\label{rref}
\end{eqnarray}
%
where
%
\begin{equation}
D = (t_1-t_2)(t_1-t_3)(t_2-t_3).
\end{equation}

%--------------------------------------------------------------

%\pagebreak
\subsection*{Problem 3a (1.0). Aster 1-3 modified.}

Consider the borehole vertical seismic profile problem of Examples 1.3 and 1.9. Consider the continuous version of the problem, represented by Eq. (1.20):
%
\begin{equation}
t(z) = \int_0^z s(z')\,dz'
\label{tz}
\end{equation}

\begin{enumerate}

\item (0.8) Assume a seismic velocity model having a linear depth gradient specified by
%
\begin{equation}
v(z) = v_0 + kz.
\label{vz}
\end{equation}
%
Solve for the analytical expression $t(z)$, recalling that $s = 1/v$.
%Recall that $s = 1/v$, and note that $\xi$ in (1.20) is a dummy integration variable for depth.

Note: The term ``velocity'' is used loosely here. It is probably better to think of ``speed'' instead. Note that $v > 0$ is a property for elastic materials.

%-----

\item (0.2) Given a set of arrival times $t(z)$, what is $s(z)$?

Hint: fundamental theorem of calculus.

\end{enumerate}

%--------------------------------------------------------------

%\pagebreak
\subsection*{Problem 3b (3.0). Aster 1-3 modified.}

Consider the borehole vertical seismic profile problem of Examples 1.3 and 1.9. Here we consider the discretized version of the problem. 

For this problem we need some numbers and a computer. Assume the following:
%
\begin{itemize}
\item There are $\ndata=100$ equally spaced seismic sensors located at depths of $z = 0.2, 0.4, \ldots, 20$~m.
\item The vertical slowness profile is represented by $\nparm=\ndata=100$ corresponding vertical intervals with midpoints at $z - 0.1$~m. Within each 0.2~m interval the slowness is assumed to be constant.
\item Let $v_0 = 1$~km/s be the velocity at the surface ($z=0$). Let the gradient in \refeq{vz} be $k = 40$~m/s per m.
\end{itemize}

\begin{enumerate}

\item (0.0) Using your formula for $t(z)$ from the previous problem, compute \textcolor{red}{a noiseless synthetic data vector $\bd^{\rm exact}$} of predicted seismic travel times. Althought we think of $t(z)$ as continuous, some discretization is needed for plotting; consider using 1000 or so $z$-points. 
{\bf (No need to list any values or expressions for this part.)}

\item (0.3)  Make two qualitative sketches like Figure 1.10, one containing model variables, the other containing data variable. Each sketch should include a coordinate axis and {\bf discretization points}. In either of the plots, include the labels $z$, $\Delta z$, $s_1$, $s_2$, $s_\nparm$, $t_1$, $t_2$, $t_\ndata$.

Note: Do not list any numbers or draw the sketch to scale.

%-----

\item (0.3) Write the discretized form of \refEq{tz}, with $s(z)$ {\bf assumed to be unknown}.

Hints: Discretize the integration increment $dz$ as $\Delta z$ (unexpanded). Avoid using the Heaviside function.

%-----

\item (0.5) Referring to your discretized version of \refEq{tz}, write a set of equations for the travel times from the surface to the $i$th sensor (at the bottom of the $i$th interval):
%
\begin{eqnarray*}
t_1 &=& ???
\\
t_2 &=& ???
\\
\vdots
\\
t_\ndata &=& ???
\end{eqnarray*}

{\bf No numbers (besides indices) should appear in your expressions.}

%-----

\item Write this system of equations in succinct form $\bd = \bG\bem$, and list the expressions for $\bd$, (matrix) $\bG$, and $\bem$.

{\bf No numbers (besides indices) should appear in your expressions.}

%-----

\item (0.4) Using \refEq{vz}, calculate the true slowness values $\bem_{\rm true}$ at the midpoints of the $n$ intervals. (No need to show expressions or values.) Compute
%
\begin{equation}
\bd^{\rm true} = \bG\bem_{\rm true}
\end{equation}
%
and compare this with $\bd^{\rm exact}$ by superimposing the two curves in a plot of $t(z)$ (travel time) vs $z$ (sensor depth). (In all plots, put depth on the $x$-axis.)

\label{calc1}

%-----

\item (0.5) Using $\bd^{\rm true}$, solve for the slowness $\bem$ as a function of depth using your $\bG$ matrix and the code options in \verb+lab_linefit.ipynb+.
%
\begin{enumerate}
\item (0.2) Show your one line of code to solve.
\item (0.3) Compare your $\bem$ and $\bem_{\rm true}$ graphically by superimposing the two curves in a plot of $s(z)$ (slowness) vs $z$ (interval midpoint depth).
\end{enumerate}

\label{calc2}

%-----

\item (0.5) \ptag\ Generate a noisy travel time vector $\dvec$ by adding independent normally distributed noise with a standard deviation of 0.05~ms ($\sigma = 5 \times 10^{-5}$~s) to the elements of $\bd^{\rm exact}$:
%
\begin{equation}
\dvec = \bd^{\rm exact} + \sigma\bw
\end{equation}
%
(Hint: Python command \verb+numpy.random.randn+) Re-solve the system for $\bem$ and again compare your results graphically with $\bem_{\rm true}$. How has the model changed?

\label{calc3}

%-----

\item (0.5) \ptag\ Repeat parts \ref{calc1}, \ref{calc2}, and \ref{calc3}, but for just $\nparm=4$ sensor depths and corresponding slowness intervals. Keep the maximum depth as the same as before (20~m), and note that $\Delta z$ is no longer 0.2~m.

Is the recovery of the true model improved? Explain in terms of the condition numbers (Aster Section A.8; command \verb+numpy.linalg.cond+) of your $\bG$ matrices.

\end{enumerate}

%--------------------------------------------------------------

%\pagebreak
\subsection*{Problem 4 (3.0). Aster 1-4 modified.}

Find a journal article that discusses the solution of an inverse problem in a discipline of special interest to you. (For example, try a google scholar search such as ``ice physics inverse problem''.)
%
\begin{enumerate}
\item (0.4) What are the data?
\item (0.2) Are the data discrete or continuous?
\item (0.3) Have the authors discussed possible sources of noise in the data?
\item (1.0) What is the model? Is the model continuous or discrete?
\item (0.5) What physical laws determine the forward operator $G$?
\item (0.2) Is $G$ linear or nonlinear?
\item (0.2) Do the authors discuss any issues associated with existence, uniqueness, or instability of solutions?
\item (0.2) Turn in the first page of your article (with title, authors, abstract, etc).
\end{enumerate}

%------------------------

%\pagebreak
\subsection*{Problem} \howmuchtime\

%-------------------------------------------------------------
\nocite{Aster}

\bibliography{uaf_abbrev,uaf_main,uaf_source,uaf_carletal,uaf_alaska}

%-------------------------------------------------------------

\appendix

\pagebreak

\section{Notes on discretizing}

This is based on Section~1.4 of \citet{Aster}.

Let's consider the integration of $f(x)$ over the interval $[a,b]$:
%
\begin{equation}
F = \int_a^b f(x)\,\rmd x
\end{equation}
%
We discretize the domain into $n$ points (see below), with each point $x_j$ centered on an interval with width $\Delta x$:
%
\begin{eqnarray}
x_j &=& a + \frac{\Delta x}{2} + (j-1)\Delta x
\\
\Delta x &=& \frac{b-a}{n}
\end{eqnarray}
%
Then
%
\begin{equation}
F = \int_a^b f(x)\,\rmd x \approx \sum_{j-1}^n f_j\,\Delta x = \left( \frac{b-a}{n} \right) \sum_{j-1}^n f_j .
\end{equation}
%
The sum of the areas of the rectangles is an approximation for the true area under the curve.

\vspace{1cm}
\includegraphics[width=12cm]{discretize}


%-------------------------------------------------------------
\end{document}
%-------------------------------------------------------------

\iffalse
Consider the expression
%
\begin{equation}
F = \int_{-\infty}^\infty f(x)\,\rmd x
\end{equation}
%
Let's assume that outside some interval $[a,b]$, $f(x) = 0$:
%
\begin{eqnarray}
f(x) &=& \left\{
\begin{array}{ll} g(x) \;\;\;\;
&  a \le x \le b
\\
0 & {\rm otherwise}
\end{array}
\right.
\end{eqnarray}
%
Then 
%
\begin{equation}
F = \int_a^b g(x)\,\rmd x
\end{equation}
\fi
