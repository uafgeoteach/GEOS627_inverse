% dvips -t letter lab_newton.dvi -o lab_newton.ps ; ps2pdf lab_newton.ps
\documentclass[11pt,titlepage,fleqn]{article}

\usepackage{amsmath}
\usepackage{amssymb}
\usepackage{latexsym}
\usepackage[round]{natbib}
%\usepackage{epsfig}
\usepackage{graphicx}
\usepackage{bm}

\usepackage{url}
\usepackage{color}

% careful: there could be conflicts when using the xr package
\usepackage{hyperref}

%--------------------------------------------------------------
%       SPACING COMMANDS (Latex Companion, p. 52)
%--------------------------------------------------------------

\usepackage{setspace}    % double-space or single-space
\usepackage{xspace}

\renewcommand{\baselinestretch}{1.2}

\textwidth 460pt
\textheight 690pt
\oddsidemargin 0pt
\evensidemargin 0pt

% see Latex Companion, p. 85
\voffset     -50pt
\topmargin     0pt
\headsep      20pt
\headheight   15pt
\headheight    0pt
\footskip     30pt
\hoffset       0pt

%\input{carlcommands}
\input{commands_letters}
\input{commands_uaf}
\input{commands_carl}

\newcommand{\blank}{xxxx}

\newcommand{\cyear}{2027}

% provide space for students to write their solutions
\newcommand{\vertgap}{\vspace{1cm}}

\graphicspath{
  {./figures/}
}

\newcommand{\repodir}{{\tt inverse}}

\newcommand{\cltag}{GEOS 627/427: Inverse Problems and Parameter Estimation, Carl Tape}
\newcommand{\ptag}{{\bf \textcolor{magenta}{[GEOS 627]}}}
%\newcommand{\ptag}{}

\bibliographystyle{agufull08}

\newcommand{\howmuchtime}{Approximately how much time {\em outside of class and lab time} did you spend on this problem set? Feel free to suggest improvements here.}

%------------------------------------------------------

\newcommand{\Ucolor}{\textcolor{red}{\bU}}
\newcommand{\Vcolor}{\textcolor{blue}{\bV}}

\newcommand{\Gcolor}{\textcolor{red}{\bU}\bS\textcolor{blue}{\bV^T}}
\newcommand{\Gpcolor}{\textcolor{red}{\bU_p}\,\bS_p\textcolor{blue}{\bV_p^T}}
\newcommand{\Gdcolor}{\textcolor{blue}{\bV_p}\,\bS_p^{-1}\textcolor{red}{\bU_p^T}}
\newcommand{\GcolorT}{\textcolor{blue}{\bV}\bS^T\textcolor{red}{\bU^T}}
\newcommand{\GpcolorT}{\textcolor{blue}{\bV_p}\,\bS_p\textcolor{red}{\bU_p^T}}
\newcommand{\GdcolorT}{\textcolor{red}{\bU_p}\,\bS_p^{-1}\textcolor{blue}{\bV_p^T}}

%------------------------------------------------------


%\pagestyle{empty}

\renewcommand{\vertgap}{\vspace{1.5cm}}

% change the figures to ``Figure L3'', etc
\renewcommand{\thefigure}{L\arabic{figure}}
\renewcommand{\thetable}{L\arabic{table}}
\renewcommand{\theequation}{L\arabic{equation}}
\renewcommand{\thesection}{L\arabic{section}}

%--------------------------------------------------------------
\begin{document}
%-------------------------------------------------------------

\begin{spacing}{1.2}
\centering
{\large \bf Lab Exercise: Iterative Newton method [newton]} \\
GEOS 626/426: Applied Seismology, Carl Tape \\
GEOS 627/427: Inverse Problems and Parameter Estimation, Carl Tape \\
%Assigned: February 9, 2012 --- Due: February 21, 2012
Last compiled: \today
\end{spacing}

%------------------------

\subsection*{Overview}

See Wikipedia ``Newton's method in optimization'': \\
\url{https://en.wikipedia.org/wiki/Newton%27s_method_in_optimization}

The quasi-Newton algorithm is (\eg Tarantola, 2005, Eq.~6.291)
%
\begin{eqnarray}
\bem_{n+1} &=& \bem_n + \bDelta\bem_n
\\
\bDelta\bem_n &=& -\bHh_n^{-1}\; \bgammah_n = -\bH_n^{-1}\; \bgamma_n
\end{eqnarray}

%------------------------

\subsection*{Problems}

\begin{enumerate}
\item Consider the equation $\bH\bem = \bg$. \\
If $\bH$ is invertible, what is the solution for $\bem$? \\
If $\bH$ is a scalar, what is the solution for $Hm = g$? (Note \footnote{In Matlab, the commands \texttt{6}\textbackslash\texttt{2} (the backslash operator \textbackslash\ is a pseudoinverse), \texttt{inv(6)*2}, and \texttt{1/6*2} are all \texttt{3}.})

%(Type \verb+LA.inv?+ for details. In fact, it won't work with a scalar.)

\vertgap

\item Open \verb+lab_newton.ipynb+. What are \verb+F+, \verb+g+, and \verb+H+?

\vertgap

%\item Using the Matlab built-in function \verb+fminbnd+, compute the (numerical) minimum of $F(\bem)$.
\item Using the Python scipy function \verb+optimize.fminbound+, compute the (numerical) minimum of $F(\bem)$.

Plot the point $(\bem_{\rm min},\,F(\bem_{\rm min}))$ on the curve $F(\bem)$.

Zoom in to show that this appears to be a true minimum.

\item By hand, sketch your setup for this problem: axes, a curve, your starting point $(\bem_0,\,F(\bem_0))$, your (known) minimum point $(\bem_{\rm min},\,F(\bem_{\rm min}))$. Describe in words how the Newton algorithm will find the minimum of $F(\bem)$. Annotate your sketch with some iterates of $\bem_n$.

Bonus: Check out the Wikipedia description, and see if you can annotate your sketch to show how the next model is obtained.

\vspace{10cm}

\item Implement the Newton algorithm, and demonstrate (graphically or numerically) that you reach the minimum.

Plot the point $(\bem,\,F(\bem))$ for each new value of $\bem$.

Try several different starting values of $\bem$ to make sure it works.

\vertgap

\item Define a sensible stopping criterion.

\vertgap

\item Is your $F(\bem_{\rm min})$ lower or higher than Python's?

\vertgap

\item Repeat the experiment for a {\bf quadratic function}. What happens?

\item For the {\bf quadratic function}, solve for the analytical minimum $\bem_{\rm min}$ and also $F(\bem_{\rm min})$. Start by writing out $F(m)$ in simplest form.

\vspace{6cm}

\item Show mathematically that the Newton algorithm will give the solution (still for the quadratic function).

\end{enumerate}

%-------------------------------------------------------------
%\bibliography{}
%-------------------------------------------------------------
\end{document}
%-------------------------------------------------------------
