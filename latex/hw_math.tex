% dvips -t letter hw_math.dvi -o hw_math.ps ; ps2pdf hw_math.ps
\documentclass[11pt,titlepage,fleqn]{article}

\usepackage{amsmath}
\usepackage{amssymb}
\usepackage{latexsym}
\usepackage[round]{natbib}
%\usepackage{epsfig}
\usepackage{graphicx}
\usepackage{bm}

\usepackage{url}
\usepackage{color}

% careful: there could be conflicts when using the xr package
\usepackage{hyperref}

%--------------------------------------------------------------
%       SPACING COMMANDS (Latex Companion, p. 52)
%--------------------------------------------------------------

\usepackage{setspace}    % double-space or single-space
\usepackage{xspace}

\renewcommand{\baselinestretch}{1.2}

\textwidth 460pt
\textheight 690pt
\oddsidemargin 0pt
\evensidemargin 0pt

% see Latex Companion, p. 85
\voffset     -50pt
\topmargin     0pt
\headsep      20pt
\headheight   15pt
\headheight    0pt
\footskip     30pt
\hoffset       0pt

%\input{carlcommands}
\input{commands_letters}
\input{commands_uaf}
\input{commands_carl}

\newcommand{\blank}{xxxx}

\newcommand{\cyear}{2027}

% provide space for students to write their solutions
\newcommand{\vertgap}{\vspace{1cm}}

\graphicspath{
  {./figures/}
}

\newcommand{\repodir}{{\tt inverse}}

\newcommand{\cltag}{GEOS 627/427: Inverse Problems and Parameter Estimation, Carl Tape}
\newcommand{\ptag}{{\bf \textcolor{magenta}{[GEOS 627]}}}
%\newcommand{\ptag}{}

\bibliographystyle{agufull08}

\newcommand{\howmuchtime}{Approximately how much time {\em outside of class and lab time} did you spend on this problem set? Feel free to suggest improvements here.}

%------------------------------------------------------

\newcommand{\Ucolor}{\textcolor{red}{\bU}}
\newcommand{\Vcolor}{\textcolor{blue}{\bV}}

\newcommand{\Gcolor}{\textcolor{red}{\bU}\bS\textcolor{blue}{\bV^T}}
\newcommand{\Gpcolor}{\textcolor{red}{\bU_p}\,\bS_p\textcolor{blue}{\bV_p^T}}
\newcommand{\Gdcolor}{\textcolor{blue}{\bV_p}\,\bS_p^{-1}\textcolor{red}{\bU_p^T}}
\newcommand{\GcolorT}{\textcolor{blue}{\bV}\bS^T\textcolor{red}{\bU^T}}
\newcommand{\GpcolorT}{\textcolor{blue}{\bV_p}\,\bS_p\textcolor{red}{\bU_p^T}}
\newcommand{\GdcolorT}{\textcolor{red}{\bU_p}\,\bS_p^{-1}\textcolor{blue}{\bV_p^T}}

%------------------------------------------------------


%--------------------------------------------------------------
\begin{document}
%-------------------------------------------------------------

\begin{spacing}{1.2}
\centering
{\large \bf Problem Set 1: Linear algebra review [math]} \\
%GEOS 626: Applied Seismology, Carl Tape
\cltag\ \\
Assigned: January 10, \cyear\ --- Due: January 24, \cyear\ \\
Last compiled: \today
\end{spacing}

%------------------------

\subsection*{Overview}

\begin{itemize}

\item The purposes of this problem set are:
%
\begin{enumerate}
\item  to review some linear algebra concepts that will be needed throughout the course.
For many problems you are welcome to use Python or Matlab to check your answers. If you find yourself doing horrendous algebra, then probably you have made a prior mistake.
\item to practice using Python for scientific computing and plotting (see \verb+hw_math.ipynb+).
If you are new to Python, let me know.
\end{enumerate}

\item Reading: Appendix A of \citet{Aster}

\item Log into OpenSARlab and run the template notebook \verb+hw_math.ipynb+. You can work directly in this file.

\item {\bf Written responses:}
Most of the answers will be done by hand. Please scan your answers and upload them as a single pdf to google drive.
%Complete as many answers as possible from inside the Jupyter notebook. Other responses, such as equation derivations (if needed), can be done with pencil and paper, then scanned and concatenated to your pdf that you upload to google drive.

\item {\bf Latex option:}
If you want to use Latex for your homeworks, see the templates in the folder \verb+latex+ (and see the \verb+README+).

\end{itemize}

%------------------------

\subsection*{Problem 1 (5.0). Matrix decompositions}

This problem must be done by hand, unless specified. ``By hand'' means showing your steps and not using decimal notation; for example, write the expression $(4 +\sqrt{3})/5$ rather than 1.1464.
Be as clear as possible in explaining what equations you are using.
%You are welcome to check your answers using Matlab.

\medskip\noindent
The matrix for this problem is
%
\begin{equation*}
\bA =  \left[ \begin{array}{rrr}
     1  &   1  &   1 \\
     3  &   2  &   4 \\
     0  &   0  &  -1 \\
\end{array} \right]
\end{equation*}

\input{hw_math_input}

%-------------------------------------------------------------

\subsection*{Problem 2 (5.0). Aster, Appendix A}

See the exercises in \citet[][Section A.12]{Aster}. \\ 
Some responses will be short: a couple sentences and equations. \\
{\bf Feel free to use a computer for row reduction (no need to show steps).}

\begin{enumerate}
\item (0.5) A.1. Let $\bA$ be an $m \times n$ matrix with $n$ pivot columns in its RREF. \\
Can the system of equations $\bA\bx=\bb$ have infinitely many solutions? Explain.

\item (0.4) A.2. If $\bC = \bA\bB$ is a $5 \times 4$ matrix, then how many rows does $\bA$ have? \\
How many columns does $\bB$ have? \\
Can you say anything about the number of columns in $\bA$?

\item (0.5) A.3. Suppose that $\bv_1$, $\bv_2$, and $\bv_3$ are three vectors in \rspace{3} and that $\bv_3 = -2\bv_1 + 3\bv_2$. Are the vectors linearly dependent or linearly independent?

\item (1.6) A.4. Let
%
\begin{equation*}
\bA =  \left[ \begin{array}{rrrr}
     1  &   2  &   3 & 4 \\
     2  &   2  &   1 & 3 \\
     4  &   6  &   7 & 11 \\
\end{array} \right]
\end{equation*}
%
Find bases for (a)~$N(\bA)$, (b)~$R(\bA)$, (c)~$N(\bA^T)$, and (d)~$R(\bA^T)$. \\
What are the dimensions of the four subspaces?

\item (0.5) A.5. Let $\bA$ be an $n \times n$ matrix such that $\bA^{-1}$ exists. \\
What are (a)~$N(\bA)$, (b)~$N(\bA^T)$, (c)~$R(\bA)$, and (d)~$R(\bA^T)$?

Hint: Use $\bA\bA^{-1} = \bA^{-1}\bA = \bI_n$.

\item (0.5) A.6. Let $\bA$ be any $9 \times 6$ matrix. If the dimension of the null space of $\bA$ is 5, then what is the dimension of $R(\bA)$? What is the dimension of $R(\bA^T)$? What is the rank of $\bA$?

\item (0.5) A.7. Suppose that a nonhomogeneous linear system of equations with four equations and six unknowns has a solution with two free variables. Is it possible to change the right-hand side of the system of equations so that the modified system of equations has no solutions?

\item (0.5) \ptag\ A.8. Let $W$ be the set of vectors $\bx = (x_1,x_2,x_3,x_4)$ in \rspace{4} such that $x_1 x_2 = 0$. (So $W$ is made up of vectors $\bx$, each having a certain property.) Is $W$ a subspace of \rspace{4}? Explain.

Hint: Come up with two vectors that violate the subspace rules.
\end{enumerate}

\subsection*{Problem} \howmuchtime\

%-------------------------------------------------------------
\bibliography{uaf_abbrev,uaf_main,uaf_carletal}
%-------------------------------------------------------------

\pagebreak
\vspace{1cm}
\input{notes_matrix_figs}

%-------------------------------------------------------------
\end{document}
%-------------------------------------------------------------
