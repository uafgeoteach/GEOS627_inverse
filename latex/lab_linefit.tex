% dvips -t letter lab_linefit.dvi -o lab_linefit.ps ; ps2pdf lab_linefit.ps
\documentclass[11pt,titlepage,fleqn]{article}

\usepackage{amsmath}
\usepackage{amssymb}
\usepackage{latexsym}
\usepackage[round]{natbib}
%\usepackage{epsfig}
\usepackage{graphicx}
\usepackage{bm}

\usepackage{url}
\usepackage{color}

% careful: there could be conflicts when using the xr package
\usepackage{hyperref}

%--------------------------------------------------------------
%       SPACING COMMANDS (Latex Companion, p. 52)
%--------------------------------------------------------------

\usepackage{setspace}    % double-space or single-space
\usepackage{xspace}

\renewcommand{\baselinestretch}{1.2}

\textwidth 460pt
\textheight 690pt
\oddsidemargin 0pt
\evensidemargin 0pt

% see Latex Companion, p. 85
\voffset     -50pt
\topmargin     0pt
\headsep      20pt
\headheight   15pt
\headheight    0pt
\footskip     30pt
\hoffset       0pt

%\input{carlcommands}
\input{commands_letters}
\input{commands_uaf}
\input{commands_carl}

\newcommand{\blank}{xxxx}

\newcommand{\cyear}{2027}

% provide space for students to write their solutions
\newcommand{\vertgap}{\vspace{1cm}}

\graphicspath{
  {./figures/}
}

\newcommand{\repodir}{{\tt inverse}}

\newcommand{\cltag}{GEOS 627/427: Inverse Problems and Parameter Estimation, Carl Tape}
\newcommand{\ptag}{{\bf \textcolor{magenta}{[GEOS 627]}}}
%\newcommand{\ptag}{}

\bibliographystyle{agufull08}

\newcommand{\howmuchtime}{Approximately how much time {\em outside of class and lab time} did you spend on this problem set? Feel free to suggest improvements here.}

%------------------------------------------------------

\newcommand{\Ucolor}{\textcolor{red}{\bU}}
\newcommand{\Vcolor}{\textcolor{blue}{\bV}}

\newcommand{\Gcolor}{\textcolor{red}{\bU}\bS\textcolor{blue}{\bV^T}}
\newcommand{\Gpcolor}{\textcolor{red}{\bU_p}\,\bS_p\textcolor{blue}{\bV_p^T}}
\newcommand{\Gdcolor}{\textcolor{blue}{\bV_p}\,\bS_p^{-1}\textcolor{red}{\bU_p^T}}
\newcommand{\GcolorT}{\textcolor{blue}{\bV}\bS^T\textcolor{red}{\bU^T}}
\newcommand{\GpcolorT}{\textcolor{blue}{\bV_p}\,\bS_p\textcolor{red}{\bU_p^T}}
\newcommand{\GdcolorT}{\textcolor{red}{\bU_p}\,\bS_p^{-1}\textcolor{blue}{\bV_p^T}}

%------------------------------------------------------


\renewcommand{\baselinestretch}{1.0}

% change the figures to ``Figure L3'', etc
\renewcommand{\thefigure}{L\arabic{figure}}
\renewcommand{\thetable}{L\arabic{table}}
\renewcommand{\theequation}{L\arabic{equation}}
\renewcommand{\thesection}{L\arabic{section}}

%--------------------------------------------------------------
\begin{document}
%-------------------------------------------------------------

\begin{spacing}{1.2}
\centering
{\large \bf Lab Exercise: Fitting a line to scattered data [linefit]} \\
GEOS 626/426: Applied Seismology, Carl Tape \\
GEOS 627/427: Inverse Problems and Parameter Estimation, Carl Tape \\
Last compiled: \today
\end{spacing}

%------------------------

\vspace{-0.5cm}
\subsection*{Overview}

The objective of this lab is to introduce the concepts of forward and inverse problems by exploring the example of fitting a line to scattered data. We also want to develop some basic capabilities in Python, such as vector-matrix operations and plotting.

\begin{enumerate}
\item Consider the forward model $d_i = m_1 + m_2 x_i$, which can be written in schematic matrix form as $\bd = \bG\bem$. If $\bd$ is $\ndata \times 1$ and $\bem$ is $2 \times 1$, what is the dimension of $\bG$?

\vertgap

\item Write down the forward model $d_i = m_1 + m_2 x_i$ in schematic matrix form $\bd = \bG\bem$.

\vertgap
\vertgap
\vertgap
\vertgap
\vertgap
\vertgap


\item Show that minimizing the least squares misfit function\footnote{Other terms for {\em misfit function} include {\em optimization function}, {\em minmization function}, {\em objective function}, {\em cost function}, and {\em loss function}. } (\ie sum of squares of residuals) leads to the solution $\bem = (\bG^T\bG)^{-1}\bG^T\dvec$ (see \verb+notes_taylor.pdf+).

\pagebreak
\item Run the script \verb+lab_linefit.ipynb+ and identify the key parts of the code associated with the forward problem and the inverse problem. What is the syntax for $\bem^T$, $\bG\bem$, and $\bG^T\bd$?

\vertgap

\item How are the ``target'' data generated? \\
How are the ``true'' data generated?


\vertgap

\item What are the different ways to compute the solution vector $\bem$? \\
To see the help for a function such as \verb+inv+, type \verb+help(la.inv)+ or \verb+la.inv?+

\vertgap

\item How does the estimated model $\bem_{\rm est}$ compare with the target model $\bem_{\rm tar}$?

\vertgap

How does the standard deviation of the residuals compare with the assumed value of $\sigma$?

\vertgap

\item What does the histogram show? \\
How does it change if you increase the number of points to $10^5$? \\
How does it change if you set $\sigma = 0$?

\vertgap

\item Re-set \verb+ndata=50+ and $\sigma = 0.3$ and add one outlier to the simulated observations. This can be done by replacing the first observation point with a value that is $1000\sigma$ larger. What happens (and why)?

(You will need to turn off the axes limits to see the outlier.)

\vertgap

\pagebreak
\item Now comment out the outlier, then examine the last block of code. Make sure you understand what is being plotted and how it is calculated.
%
\begin{itemize}
\item What is the dimension of model space?
\item Fill out the table below. List ``scalar'' if the variable is a scalar.
%\item XXX In the code, why is the operation \verb+res.*res+ used instead of \verb+res*res+, \verb+res'*res+, or \verb+res*res'+? XXX
\end{itemize}

\begin{tabular}{c|c|c|c}
\hline
variable & code variable   & dimension      & description \\ \hline\hline
         & \verb+ndata+    &                &              \\ \hline
         & \verb+nparm+    &                &              \\ \hline
         & \verb+dobs+     &                &              \\ \hline
         & \verb+G+        &                &              \\ \hline
$\misfitvar(\bem)$  &      &                &              \\ \hline
         & \verb+ngrid+    &                &              \\ \hline
---      & \verb+RSSm+     &                &              \\ \hline
\bem     & \verb+mtry+     & $2 \times 1$   & \hspace{3cm} \\ \hline
         & \verb+dtry+     &                &              \\ \hline
         & \verb+res+      &                &              \\ \hline
\bgamma(\bem)  & --        & $2 \times 1$   & gradient of misfit function \\ \hline
--       & \verb+gammam+   & $2 \times n_g$ &  \\ \hline
\end{tabular}

%==============================================================
% GEOS 626 STOP HERE THE FIRST TIME THROUGH
%==============================================================

\pagebreak

\item Using your answer from Problem~3, compute the gradient $\bgamma(\bem)$ for the same grid of $\bem$ that was used to plot the misfit function.

Hint: one line of code (plus one line to initialize $\bgamma(\bem)$)

The \verb+gradient+ function will not help you here, since you need an exact evaluation of the gradient, whereas \verb+gradient+ will provide a numerical gradient for a grid of pre-computed values.

\vertgap

\item Using the \verb+quiver+ command, superimpose the vector field $-\gamma(\bem)$ on the contour plot of the misfit function.

Set \verb+nx = 10+ so that the vector field is more visible. \\
Hint: This is one line of code. \\
What is the relationship between the contours of $F(\bem)$ and $\bgamma(\bem)$?

\vertgap

How does $\|\bgamma(\bem)\|$ vary with respect to $\bem_{\rm est}$? \\
% it gets larger 
Bonus: Use \verb+plot_surface+ to plot $\|\bgamma(\bem)\|$. What shape is it?
% a cone

\vertgap

\item What is the dependence of the Hessian $\bH(\bem)$ on the model? \\
% Hessian does not depend on m

\vertgap

What does that imply about $F(\bem)$?
% it is strictly quadratic

%What does that imply about the forward model $\bg(\bem)$?

\end{enumerate}

%-------------------------------------------------------------
\end{document}
%-------------------------------------------------------------
